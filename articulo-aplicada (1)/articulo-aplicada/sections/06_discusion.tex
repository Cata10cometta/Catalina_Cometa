\section{Discusión}
La experiencia con el proyecto “Experiencias Significativas” confirma que aplicar principios de Arquitectura de Software aporta beneficios tangibles en proyectos educativos y de gestión.

	extbf{Ventajas:}
\begin{itemize}
	\item La organización por capas permitió separar responsabilidades y facilitó el trabajo colaborativo.
	\item El uso de servicios y componentes hizo posible la integración de nuevas funcionalidades sin grandes cambios en el sistema.
	\item La claridad en la estructura redujo el tiempo de resolución de problemas y mejoró la calidad general del software.
\end{itemize}

	extbf{Limitaciones y retos:}
\begin{itemize}
	\item La implementación inicial requiere mayor planificación y diseño.
	\item Es necesario que todo el equipo comprenda y respete la estructura definida para evitar desorden a futuro.
\end{itemize}

En conclusión, la arquitectura aplicada no solo mejoró el desarrollo y mantenimiento, sino que también sentó las bases para la evolución futura del sistema. Se recomienda a otros equipos considerar estos principios desde el inicio de sus proyectos para obtener resultados similares.