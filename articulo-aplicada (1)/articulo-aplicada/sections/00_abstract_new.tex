\textbf{Resumen}---Este artículo se centra en un análisis detallado de veinte artículos científicos que nos hablan sobre los patrones de diseño, las arquitecturas más contemporáneas y los pilares del desarrollo de software. Y es que, de hecho, todo este estudio no fue en vano: se convirtió en la base teórica directa para el diseño de nuestra Plataforma de Gestión de Experiencias Significativas. Se analizan los patrones de comportamiento, estructurales y creacionales, además de arquitecturas como son Domain-Driven Design (DDD), N-Capas, Onion Architecture y MVC. Esto es crucial, ya que el proyecto se fundamenta en la arquitectura N-Capas para su backend (implementado en C\#/.NET Core 8 con Entity Framework Core, SignalR y JWT Bearer) y evoluciona hacia DDD y Microservicios para garantizar la escalabilidad.

Asimismo, tratan la seguridad del software, la automatización de pruebas, los microservicios, las tendencias futuras y los principios SOLID. Los patrones de diseño siguen siendo instrumentos primordiales para asegurar la escalabilidad, la calidad y el mantenimiento en sistemas complejos. También queda claro que su uso apropiado ayuda a crear un software seguro y reutilizable. En esencia, el proyecto es nuestro caso de prueba para validar que esta teoría funciona en la realidad del sector educativo.

\textbf{Abstract}---This article centers on a detailed analysis of twenty scientific papers that discuss design patterns, contemporary architectures, and the pillars of software development. And, as it turns out, this entire study wasn't done in vain: it became the direct theoretical foundation for the design of our Significant Experiences Management Platform. The paper analyzes creational, structural, and behavioral patterns, in addition to architectures such as Domain-Driven Design (DDD), N-Tier, Onion Architecture, and MVC. This is crucial, as the project's backend is fundamentally built upon the N-Tier architecture (implemented in C\#/.NET Core 8 with Entity Framework Core, SignalR, and JWT Bearer) and evolves toward DDD and Microservices to ensure scalability.

Furthermore, the review covers software security, test automation, microservices, future trends, and the SOLID principles. Design patterns remain essential instruments for ensuring scalability, quality, and maintenance in complex systems. It's also clear that their appropriate use helps create software that is secure and reusable. In essence, the project serves as our practical case study to validate that this theory functions within the realities of the education sector.
