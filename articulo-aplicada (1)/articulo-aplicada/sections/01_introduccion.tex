La Arquitectura de Software va muchisimo mas alla de solo escribir codigo. No, esto implica organizar, estructurar, y coordinar los componentes del sistema, buscando un correcto funcionamiento seguro y que pueda escalarse. Con una arquitectura bien pensada, los elementos internos pueden comunicarse, evitando las dependencias que sobran y asegurando el crecimiento de la aplicación sin problemas de estabilidad.

En el proyecto “Experiencias Significativas”, la arquitectura fue clave, desde la etapa inicial de diseño. Esta plataforma, pensada como una herramienta tecnológica para los profes de una institución, les ayuda a organizar, administrar y evaluar las experiencias formativas y administrativas, con claridad y de forma estandarizada.

El sistema, ademas de facilitar el registro de experiencias educativas, también permite su valoracion y visualización con interfaces simples, intuitivas y accesibles, todo para que la gente entienda. Para lograrlo, se usaron principios fundamentales como estos:

Arquitectura por capas, separando responsabilidades.

Uso de servicios, para la interacción entre componentes, ¿entiendes?

Diseño orientado a componentes, para el frontend, ya sabes.

Estructuras modulares, lo cual es muy util para el mantenimiento.

Consumo de APIs REST, el intercambio organizado de información.
De verdad, cada una de estas elecciones ayudó a que la plataforma tuviera una buena organización interna, que escalara, y se notará su estructura.