\section{Introducción}
La Arquitectura de Software es esencial en el proceso de sistemas. No se trata solo de líneas de código, sino de cómo se organizan las piezas, cómo se comunican entre sí y cómo todo esto permite que una aplicación funcione de manera segura, confiable y escalable.

El proyecto “Experiencias Significativas” está pensado para ayudar a los docentes de una institución educativa a organizar y evaluar sus actividades formativas y administrativas. Para su construcción se aplicaron conceptos fundamentales de la Arquitectura de Software, como la organización por capas, el uso de servicios y el diseño orientado a componentes.

Se aplicaron principios fundamentales:
\begin{itemize}
	\item \textbf{Organización por capas}: permite separar responsabilidades y tener un orden.
	\item \textbf{Uso de Servicios}: facilita la comunicación entre componentes.
	\item \textbf{Diseño orientado a componentes}: asegura flexibilidad y reutilización.
\end{itemize}