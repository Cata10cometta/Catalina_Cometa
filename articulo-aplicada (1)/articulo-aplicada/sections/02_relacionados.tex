\section{Trabajos relacionados}
La literatura sobre Arquitectura de Software ha establecido principios fundamentales para la construcción de sistemas robustos y mantenibles.

Kruchten \cite{kruchten1995architectural} propone el modelo 4+1 vistas, estableciendo un marco conceptual para documentar y comunicar la arquitectura de sistemas complejos. Cardacci \cite{cardacci2015arquitectura} presenta una arquitectura académica para la comprensión del desarrollo por capas, destacando la importancia de la separación de responsabilidades.

Fernández \cite{fernandez2006arquitectura} discute los fundamentos de la arquitectura de software y su impacto en la calidad del producto final. Jimenez-Torres et al. \cite{jimenez2014lenguajes} realizan una aproximación al estado del arte de los lenguajes de patrones arquitectónicos, proporcionando un marco de referencia para la toma de decisiones.

Navarro et al. \cite{navarro2017integracion} abordan la integración de la arquitectura de software en metodologías ágiles, mientras que Romero \cite{romero2006arquitectura} analiza esquemas y servicios en la arquitectura moderna. Vera \cite{vera2023arquitectura} explora la relación entre arquitectura de software y programación orientada a objetos, y Cambarieri et al. \cite{cambarieri2020implementacion} presentan una implementación guiada por el dominio.