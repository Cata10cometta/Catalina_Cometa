\section{Implementación del software}
El sistema "Experiencias Significativas" se implementó siguiendo una arquitectura por capas que separa las responsabilidades y facilita el mantenimiento del código.

\subsection{Arquitectura del sistema}
La arquitectura implementada organiza el código en capas claramente definidas: controladores para manejar las peticiones, servicios para la lógica de negocio, repositorios para el acceso a datos, y entidades para representar los modelos del dominio.

\subsection{Fragmento de código}
Ejemplo de implementación de una función con tipado estático:

\ifminted
\begin{minted}[linenos]{python}
def suma(a: int, b: int) -> int:
    """Suma dos numeros enteros.
    
    Args:
        a: Primer operando
        b: Segundo operando
        
    Returns:
        La suma de a y b
    """
    return a + b
\end{minted}
\else
\begin{lstlisting}[language=Python]
def suma(a: int, b: int) -> int:
    """Suma dos numeros enteros.
    
    Args:
        a: Primer operando
        b: Segundo operando
        
    Returns:
        La suma de a y b
    """
    return a + b
\end{lstlisting}
\fi

\subsection{Tecnologías utilizadas}
La selección de tecnologías se basó en criterios de escalabilidad, mantenibilidad y facilidad de uso, priorizando herramientas que facilitaran la implementación de la arquitectura por capas.